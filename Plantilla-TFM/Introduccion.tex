
\initial{T}transport problems are of increasing importance as industry and service activities grows. Computational methods to find optimal or near optimal routes could represent savings. That is the reason why \textbf{companies are interested in developing applications which automate routes design in order to minimize their cost}. Transportation management can be usually defined by a pattern where a centralized fleet of vehicles distributes goods from a depot to a set of customers in different locations. As the number of customers, vehicles, deliveries or goods, grow finding a good routing plan turns into a difficult task.

This real situation can be modeled by vehicle routing problems. The vehicle routing problem, VRP, which was first introduced by Dantzing and Ramser, \cite{Dantzig}, can represent those real situations. \textbf{The VRP describes a real-life problem and introduces a mathematical programming formulation}. Several variation and specializations of the vehicle routing problem exist. This work faces the vehicle routing problem with time windows. 

The objective of the VRPTW is to supply a number of customers within predefined time windows at minimum cost (commonly, in terms of traveled distance), respecting capacity constraints for each vehicle of a homogeneous fleet, i.e., given a number of customers with demands and time to be visited, a homogeneous fleet with capacities, \textbf{the VRPTW consists of finding a configuration of routes starting and ending at a depot visiting every customer once in time}. Each customer specifies the earliest and latest time for the start of the service, this requirement defines time window constraints.

A standard objective of the VRPTW aims to minimize the number of routes or vehicles, primary criterion, and the total travel costs, secondary criterion. However, other objective functions have been considered in various papers.
\textbf{The VRPTW is non-polynomial-hard, NP-complete} \cite{Lenstra}, thus most of real problems cannot be solved within hours using exact models. Also, some large theorical benchmarks are hard to solve to optimality. Indeed, very few of the Solomon benchmark instances \cite{Solomon_1987} involving 100 customers have been solved optimally \cite{Fisher}.
As a consequence, \textbf{heuristic or metaheuristic algorithms}, such as tabu search, genetic algorithm, evolutionary algorithms and ant colony optimization algorithm can offer a good approach to the problem.

This Master Thesis \textbf{reviews the VRPWT literature}, focusing in metaheuristics applied in medium size instances, up to 100 customers. It proposes a metaheuristic which combines a randomized 2-optimization insertion and swapping algorithm for this transportation problem. A biased random sampling component is introduced to transform a deterministic heuristic into a probabilistic algorithm, as well as guides the local search. Two different initial solutions are considered: a dummy solution and a different one constructed by clustering customers by their locations. The algorithm starts from one initial solution to improve it by swapping or inserting nodes between routes. The metaheuristic has been tested under a theoretical benchmark and has been adapted to solve a real agricultural problem which aims to minimize its transportation costs. Experimental results prove the effectiveness of the metaheuristic, which has \textbf{matched 10 of these solutions}, and has showed good results to a real agricultural cooperative problem. 

On the other hand, the collaborative economy is rapidly emerging and the transportation sector is highly competitive. Transport companies must keep their cost low. To reduce cost, companies can collaborate, where part of their daily operation are scheduled jointly. This collaboration can called be called \textbf{collaborative vehicle routing problem, where companies cooperate to improve their efficiency in terms of fleet operation}.

The collaborative routing problem raises a main question: \textbf{how to split the benefit among the companies}. This thesis reviews cost allocation methods as a tool for allocating costs in the collaborative routing problem. Cost allocation methods splits the common cost among the participants. The most simple solution to cost allocation is splitting the common cost equally, weighted by any criteria. However, this cost allocation is probably unfair, for example, maybe its allocates a company more cost than when operating alone. 

\textbf{The thesis gives a review of the main results of the cooperation game theory}. It focus on allocation values such as \textbf{Shapley value and the nucleolus}. These values are widely used in literature for not large games, when they can be computed within a reasonable amount of time. This work faces medium size routing problem, in practice it solves instances up to 100 customers. The Shapley value is difficult to be computed to these problem. However, the \textbf{Aumann-Dr\`eze's value} is a good alternative to allocate cost in large cooperative games. Moreover, there are some well know values such as the \textbf{Equal Profit Method} (EPM) and the \textbf{Lorenz allocation}. Finally the thesis introduces these values to collaborative routing problems to a theorical benchmark up tp 100 customers.


The outline of this work is as follows. The work is structured in two main blocks: optimization (\ref{chap:blockI}) and game theory (\ref{chap:blockII}). 
Block I contains a review of the most relevant work in the VRP (\ref{intro2}) and gives an overview of the methodological approach in sections (\ref{review}). Section \ref{proposed_metaheuristic} defines the problem formulation. Section \ref{proposed_metaheuristic} presents the proposed metaheuristic, detailing each of its phases and the two different initial solutions considered.
Section \ref{results} shows the experimental results obtained by testing the metaheuristic with two different problems: a set of classical VRPTW instances with heterogeneous fleet and capacity problem (\cite{Solomon_1987}) and a real agricultural cooperative VRPTW. Furthermore, section \ref{Robustness} analyses the robustness of the algorithm. 
Block II introduces the main concepts of transferable utility games (\ref{GameTheoryBackground}) and gives an overview of the main allocation rules (\ref{AllocationRules}). Section \ref{GameTheoryCooperation} reviews the literature about collaborative routing problem. Section \ref{GameTheoryResults} presents how to apply allocate cost to one solution of the Solomon's benchmark (\ref{CooperationSolomon}). 
Finally, section \ref{chap:Conclusion} concludes the thesis summing up its main conclusions



